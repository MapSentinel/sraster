\documentclass[]{article}
\usepackage{lmodern}
\usepackage{amssymb,amsmath}
\usepackage{ifxetex,ifluatex}
\usepackage{fixltx2e} % provides \textsubscript
\ifnum 0\ifxetex 1\fi\ifluatex 1\fi=0 % if pdftex
  \usepackage[T1]{fontenc}
  \usepackage[utf8]{inputenc}
\else % if luatex or xelatex
  \ifxetex
    \usepackage{mathspec}
  \else
    \usepackage{fontspec}
  \fi
  \defaultfontfeatures{Ligatures=TeX,Scale=MatchLowercase}
\fi
% use upquote if available, for straight quotes in verbatim environments
\IfFileExists{upquote.sty}{\usepackage{upquote}}{}
% use microtype if available
\IfFileExists{microtype.sty}{%
\usepackage{microtype}
\UseMicrotypeSet[protrusion]{basicmath} % disable protrusion for tt fonts
}{}
\usepackage[margin=1in]{geometry}
\usepackage{hyperref}
\hypersetup{unicode=true,
            pdftitle={Use of global land},
            pdfauthor={DGT Portugal, William Martinez},
            pdfborder={0 0 0},
            breaklinks=true}
\urlstyle{same}  % don't use monospace font for urls
\usepackage{color}
\usepackage{fancyvrb}
\newcommand{\VerbBar}{|}
\newcommand{\VERB}{\Verb[commandchars=\\\{\}]}
\DefineVerbatimEnvironment{Highlighting}{Verbatim}{commandchars=\\\{\}}
% Add ',fontsize=\small' for more characters per line
\usepackage{framed}
\definecolor{shadecolor}{RGB}{248,248,248}
\newenvironment{Shaded}{\begin{snugshade}}{\end{snugshade}}
\newcommand{\AlertTok}[1]{\textcolor[rgb]{0.94,0.16,0.16}{#1}}
\newcommand{\AnnotationTok}[1]{\textcolor[rgb]{0.56,0.35,0.01}{\textbf{\textit{#1}}}}
\newcommand{\AttributeTok}[1]{\textcolor[rgb]{0.77,0.63,0.00}{#1}}
\newcommand{\BaseNTok}[1]{\textcolor[rgb]{0.00,0.00,0.81}{#1}}
\newcommand{\BuiltInTok}[1]{#1}
\newcommand{\CharTok}[1]{\textcolor[rgb]{0.31,0.60,0.02}{#1}}
\newcommand{\CommentTok}[1]{\textcolor[rgb]{0.56,0.35,0.01}{\textit{#1}}}
\newcommand{\CommentVarTok}[1]{\textcolor[rgb]{0.56,0.35,0.01}{\textbf{\textit{#1}}}}
\newcommand{\ConstantTok}[1]{\textcolor[rgb]{0.00,0.00,0.00}{#1}}
\newcommand{\ControlFlowTok}[1]{\textcolor[rgb]{0.13,0.29,0.53}{\textbf{#1}}}
\newcommand{\DataTypeTok}[1]{\textcolor[rgb]{0.13,0.29,0.53}{#1}}
\newcommand{\DecValTok}[1]{\textcolor[rgb]{0.00,0.00,0.81}{#1}}
\newcommand{\DocumentationTok}[1]{\textcolor[rgb]{0.56,0.35,0.01}{\textbf{\textit{#1}}}}
\newcommand{\ErrorTok}[1]{\textcolor[rgb]{0.64,0.00,0.00}{\textbf{#1}}}
\newcommand{\ExtensionTok}[1]{#1}
\newcommand{\FloatTok}[1]{\textcolor[rgb]{0.00,0.00,0.81}{#1}}
\newcommand{\FunctionTok}[1]{\textcolor[rgb]{0.00,0.00,0.00}{#1}}
\newcommand{\ImportTok}[1]{#1}
\newcommand{\InformationTok}[1]{\textcolor[rgb]{0.56,0.35,0.01}{\textbf{\textit{#1}}}}
\newcommand{\KeywordTok}[1]{\textcolor[rgb]{0.13,0.29,0.53}{\textbf{#1}}}
\newcommand{\NormalTok}[1]{#1}
\newcommand{\OperatorTok}[1]{\textcolor[rgb]{0.81,0.36,0.00}{\textbf{#1}}}
\newcommand{\OtherTok}[1]{\textcolor[rgb]{0.56,0.35,0.01}{#1}}
\newcommand{\PreprocessorTok}[1]{\textcolor[rgb]{0.56,0.35,0.01}{\textit{#1}}}
\newcommand{\RegionMarkerTok}[1]{#1}
\newcommand{\SpecialCharTok}[1]{\textcolor[rgb]{0.00,0.00,0.00}{#1}}
\newcommand{\SpecialStringTok}[1]{\textcolor[rgb]{0.31,0.60,0.02}{#1}}
\newcommand{\StringTok}[1]{\textcolor[rgb]{0.31,0.60,0.02}{#1}}
\newcommand{\VariableTok}[1]{\textcolor[rgb]{0.00,0.00,0.00}{#1}}
\newcommand{\VerbatimStringTok}[1]{\textcolor[rgb]{0.31,0.60,0.02}{#1}}
\newcommand{\WarningTok}[1]{\textcolor[rgb]{0.56,0.35,0.01}{\textbf{\textit{#1}}}}
\usepackage{graphicx,grffile}
\makeatletter
\def\maxwidth{\ifdim\Gin@nat@width>\linewidth\linewidth\else\Gin@nat@width\fi}
\def\maxheight{\ifdim\Gin@nat@height>\textheight\textheight\else\Gin@nat@height\fi}
\makeatother
% Scale images if necessary, so that they will not overflow the page
% margins by default, and it is still possible to overwrite the defaults
% using explicit options in \includegraphics[width, height, ...]{}
\setkeys{Gin}{width=\maxwidth,height=\maxheight,keepaspectratio}
\IfFileExists{parskip.sty}{%
\usepackage{parskip}
}{% else
\setlength{\parindent}{0pt}
\setlength{\parskip}{6pt plus 2pt minus 1pt}
}
\setlength{\emergencystretch}{3em}  % prevent overfull lines
\providecommand{\tightlist}{%
  \setlength{\itemsep}{0pt}\setlength{\parskip}{0pt}}
\setcounter{secnumdepth}{0}
% Redefines (sub)paragraphs to behave more like sections
\ifx\paragraph\undefined\else
\let\oldparagraph\paragraph
\renewcommand{\paragraph}[1]{\oldparagraph{#1}\mbox{}}
\fi
\ifx\subparagraph\undefined\else
\let\oldsubparagraph\subparagraph
\renewcommand{\subparagraph}[1]{\oldsubparagraph{#1}\mbox{}}
\fi

%%% Use protect on footnotes to avoid problems with footnotes in titles
\let\rmarkdownfootnote\footnote%
\def\footnote{\protect\rmarkdownfootnote}

%%% Change title format to be more compact
\usepackage{titling}

% Create subtitle command for use in maketitle
\providecommand{\subtitle}[1]{
  \posttitle{
    \begin{center}\large#1\end{center}
    }
}

\setlength{\droptitle}{-2em}

  \title{Use of global land}
    \pretitle{\vspace{\droptitle}\centering\huge}
  \posttitle{\par}
    \author{DGT Portugal, William Martinez}
    \preauthor{\centering\large\emph}
  \postauthor{\par}
      \predate{\centering\large\emph}
  \postdate{\par}
    \date{05/08/2019}


\begin{document}
\maketitle

{
\setcounter{tocdepth}{4}
\tableofcontents
}
\hypertarget{result-accuracies-per-class}{%
\subsection{Result accuracies per
class}\label{result-accuracies-per-class}}

\begin{Shaded}
\begin{Highlighting}[]
\KeywordTok{library}\NormalTok{(ggplot2)}
\NormalTok{file_comparision_models =}\StringTok{ 'C:}\CharTok{\textbackslash{}\textbackslash{}}\StringTok{IPSTERS}\CharTok{\textbackslash{}\textbackslash{}}\StringTok{sraster}\CharTok{\textbackslash{}\textbackslash{}}\StringTok{Results}\CharTok{\textbackslash{}\textbackslash{}}\StringTok{Accuraccies_perclass.csv'}

\CommentTok{#reading file}
\NormalTok{comparision_models  =}\StringTok{ }\KeywordTok{read.csv}\NormalTok{(file_comparision_models, }\DataTypeTok{sep =} \StringTok{";"}\NormalTok{,}\DataTypeTok{header =}\NormalTok{ T, }\DataTypeTok{dec =} \StringTok{','}\NormalTok{)}

\NormalTok{comparision_models}\OperatorTok{$}\NormalTok{Class =}\StringTok{ }\KeywordTok{factor}\NormalTok{(comparision_models}\OperatorTok{$}\NormalTok{Class,}\DataTypeTok{levels =} \KeywordTok{c}\NormalTok{(}\StringTok{"urban"}\NormalTok{,}\StringTok{"baresoil"}\NormalTok{, }\StringTok{"rainfed"}\NormalTok{, }\StringTok{"irrigated"}\NormalTok{, }\StringTok{"rice field"}\NormalTok{, }\StringTok{"A_grassland"}\NormalTok{, }\StringTok{"broadleaf"}\NormalTok{, }\StringTok{"conifers"}\NormalTok{, }\StringTok{"N_grassland"}\NormalTok{, }\StringTok{"shrubland"}\NormalTok{, }\StringTok{"S_vegetation"}\NormalTok{, }\StringTok{"wetland"}\NormalTok{, }\StringTok{"water"}\NormalTok{))}

\NormalTok{comparision_models}\OperatorTok{$}\NormalTok{External=}\StringTok{ }\KeywordTok{factor}\NormalTok{(comparision_models}\OperatorTok{$}\NormalTok{External , }\DataTypeTok{levels =} \KeywordTok{c}\NormalTok{(}\StringTok{'COS'}\NormalTok{,}\StringTok{'COS_HRL'}\NormalTok{,}\StringTok{'COS_GL30_HRL'}\NormalTok{))}

\NormalTok{comparision_models}\OperatorTok{$}\NormalTok{Samples <-}\StringTok{ }\KeywordTok{as.factor}\NormalTok{(comparision_models}\OperatorTok{$}\NormalTok{Samples)}

\NormalTok{comparision_models}\OperatorTok{$}\NormalTok{Polygons <-}\StringTok{ }\KeywordTok{as.factor}\NormalTok{(comparision_models}\OperatorTok{$}\NormalTok{Polygons)}

\NormalTok{comparision_models1 =}\StringTok{ }\NormalTok{comparision_models[comparision_models}\OperatorTok{$}\NormalTok{External }\OperatorTok\StringTok{ }\KeywordTok{c}\NormalTok{(}\StringTok{'COS'}\NormalTok{,}\StringTok{'COS_HRL'}\NormalTok{,}\StringTok{'COS_GL30_HRL'}\NormalTok{),]}

\CommentTok{#ggplot}
\NormalTok{p1 <-}\StringTok{ }\KeywordTok{ggplot}\NormalTok{(}\DataTypeTok{data =}\NormalTok{ comparision_models1 , }\KeywordTok{aes}\NormalTok{(}\DataTypeTok{x=}\NormalTok{Class, }\DataTypeTok{y=}\NormalTok{F1, }\DataTypeTok{fill =}\NormalTok{ External )) }\OperatorTok{+}\StringTok{ }\KeywordTok{geom_boxplot}\NormalTok{() }\OperatorTok{+}\StringTok{ }
\StringTok{  }\KeywordTok{theme}\NormalTok{(}\DataTypeTok{axis.text.x =} \KeywordTok{element_text}\NormalTok{(}\DataTypeTok{size=}\DecValTok{15}\NormalTok{, }
                                    \DataTypeTok{face=}\StringTok{"bold.italic"}\NormalTok{,}\DataTypeTok{angle =} \DecValTok{90}\NormalTok{),}
        \DataTypeTok{axis.text.y =} \KeywordTok{element_text}\NormalTok{(}\DataTypeTok{angle =} \DecValTok{0}\NormalTok{,}\DataTypeTok{size=}\DecValTok{12}\NormalTok{,}
                                      \DataTypeTok{face=}\StringTok{"bold.italic"}\NormalTok{),}
        \DataTypeTok{axis.title.y =} \KeywordTok{element_text}\NormalTok{(}\DataTypeTok{size=}\DecValTok{14}\NormalTok{, }\DataTypeTok{face=}\StringTok{"bold"}\NormalTok{),}
        \DataTypeTok{axis.title.x =} \KeywordTok{element_text}\NormalTok{(}\DataTypeTok{size=}\DecValTok{14}\NormalTok{, }\DataTypeTok{face=}\StringTok{"bold"}\NormalTok{)) }\OperatorTok{+}
\StringTok{        }\KeywordTok{expand_limits}\NormalTok{(}\DataTypeTok{y =} \KeywordTok{c}\NormalTok{(}\DecValTok{0}\NormalTok{, }\DecValTok{1}\NormalTok{)) }
\CommentTok{#+scale_fill_manual(values=c("#FC4E07", "#C4961A"))}
\KeywordTok{print}\NormalTok{(p1)}
\end{Highlighting}
\end{Shaded}

\begin{verbatim}
## Warning: Removed 1 rows containing non-finite values (stat_boxplot).
\end{verbatim}

\includegraphics{Results_files/figure-latex/unnamed-chunk-1-1.pdf}

\hypertarget{result-trayectories}{%
\subsection{Result trayectories}\label{result-trayectories}}

\begin{Shaded}
\begin{Highlighting}[]
\KeywordTok{library}\NormalTok{(ggplot2)}
\KeywordTok{library}\NormalTok{(reshape2)}
\end{Highlighting}
\end{Shaded}

\begin{verbatim}
## Warning: package 'reshape2' was built under R version 3.6.1
\end{verbatim}

\begin{Shaded}
\begin{Highlighting}[]
\NormalTok{path_file =}\StringTok{ 'C:}\CharTok{\textbackslash{}\textbackslash{}}\StringTok{IPSTERS}\CharTok{\textbackslash{}\textbackslash{}}\StringTok{sraster}\CharTok{\textbackslash{}\textbackslash{}}\StringTok{ins4}\CharTok{\textbackslash{}\textbackslash{}}\StringTok{sampling_4}\CharTok{\textbackslash{}\textbackslash{}}\StringTok{output_broadleaf2_Majorityrule_bdist.csv'}
\NormalTok{data_class =}\StringTok{ }\KeywordTok{read.csv}\NormalTok{(path_file, }\DataTypeTok{sep=} \StringTok{','}\NormalTok{, }\DataTypeTok{header =}\NormalTok{ T, }\DataTypeTok{dec =} \StringTok{'.'}\NormalTok{ )}

\NormalTok{col_ndvi =}\StringTok{ }\KeywordTok{grep}\NormalTok{(}\StringTok{"NDVI"}\NormalTok{,}\KeywordTok{colnames}\NormalTok{(data_class))}
\NormalTok{data_class_ndvi =}\StringTok{ }\NormalTok{data_class[,}\KeywordTok{c}\NormalTok{(}\DecValTok{3}\NormalTok{,col_ndvi)]}

\NormalTok{func_stat <-}\StringTok{ }\ControlFlowTok{function}\NormalTok{(l)\{}\KeywordTok{quantile}\NormalTok{(l,}\KeywordTok{c}\NormalTok{(}\FloatTok{0.5}\NormalTok{),}\DataTypeTok{na.rm=}\OtherTok{TRUE}\NormalTok{)\}}
\NormalTok{result_aggregate_time =}\StringTok{ }\KeywordTok{aggregate}\NormalTok{(data_class_ndvi, }\DataTypeTok{by =} \KeywordTok{list}\NormalTok{(data_class_ndvi}\OperatorTok{$}\NormalTok{Object), }\DataTypeTok{FUN =}\NormalTok{ func_stat)}
\NormalTok{resultmelt <-}\StringTok{ }\KeywordTok{melt}\NormalTok{(result_aggregate_time[,}\OperatorTok{-}\DecValTok{1}\NormalTok{], }\DataTypeTok{id.vars =} \StringTok{"Object"}\NormalTok{)}
\CommentTok{#removing ndvi text}
\NormalTok{resultmelt}\OperatorTok{$}\NormalTok{variable <-}\StringTok{ }\KeywordTok{as.Date}\NormalTok{(}\KeywordTok{substr}\NormalTok{(resultmelt[,}\KeywordTok{c}\NormalTok{(}\StringTok{"variable"}\NormalTok{)], }\DecValTok{2}\NormalTok{, }\DecValTok{11}\NormalTok{),}\StringTok{"%Y.%m.%d"}\NormalTok{)}
\NormalTok{resultmelt}\OperatorTok{$}\NormalTok{Object <-}\StringTok{ }\KeywordTok{as.factor}\NormalTok{(resultmelt}\OperatorTok{$}\NormalTok{Object)}

\NormalTok{p1 <-}\StringTok{ }\KeywordTok{ggplot}\NormalTok{(resultmelt, }\KeywordTok{aes}\NormalTok{(variable, value, }\DataTypeTok{group =}\NormalTok{ Object)) }\OperatorTok{+}\StringTok{ }
\StringTok{  }\KeywordTok{geom_line}\NormalTok{(}\DataTypeTok{color=}\StringTok{"red"}\NormalTok{) }\OperatorTok{+}\StringTok{ }\KeywordTok{theme}\NormalTok{(}\DataTypeTok{legend.position=}\StringTok{"top"}\NormalTok{)}
\KeywordTok{print}\NormalTok{(p1)}
\end{Highlighting}
\end{Shaded}

\includegraphics{Results_files/figure-latex/unnamed-chunk-2-1.pdf}


\end{document}
